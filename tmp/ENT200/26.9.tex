
% Preamble
% -----------------------------------------------------------------

\documentclass{article}

% Packages
% -----------------------------------------------------------------

%\usepackage{amsmath}                    % Advanced math typesetting
%\usepackage [utf8] {inputenc}       % Unicode support (Umlauts etc)
%\usepackage [ngerman] {babel}             % Change hyphenationrules
%\usepackage{hyperref}                 % Add a link to your document
%\usepackage{graphicx}                % Add picturesto your document
%\usepackage{listing}      % Source Code formatting and highlighting
%\usepackage{enumitem}                  % Used for enumeration-lists

% Document Info
% -----------------------------------------------------------------

\author{Ty Yiu}                                 % The author's name
\title{Latex Template}           % The title of the document
\date{25 September 2018}
\begin{document}
\maketitle                                        % Generates title 
%
% \tableofcontents{}                  % Generates table of contents 
                                    % from sections and subsections

% Commands
% -----------------------------------------------------------------

%\section{}                                          % Section Title

% --------------------

%\subsection{}                                    % Section Title XS

% --------------------

%\subsubsection{}                                % Section Title XXS

% --------------------

%\paragraph{}                                      % Paragraph Title

% --------------------

%\subparagraph{}                                % Subparagraph Title

% --------------------

%\begin{itemize}                                 % Unenumerated List
%\item ITEM
%\item ITEM
%\item ITEM
%\item ITEM
%\end{itemize}
 
% --------------------

%\begin{enumerate}                         % Enumerated List (1.2.3.)
%\item ITEM
%\item ITEM
%\end{enumerate}

% --------------------

%\begin{enumerate}                         % Enumerated List (1.2.3.)
%\item ITEM                    % With Nested List with Items (a.b.c.)
%\begin{enumerate}
%\item ITEM
%\item ITEM
%\end{enumerate}
%\item ITEM
%\end{enumerate}

% --------------------    

%\textit{}                                              % Italic Text

% --------------------

%\textbf{}                                                 % Bold Text

% --------------------

\section{Calculating NPV and The Payback Period}

This is what we call the initial aspect of entrepreneurial finance.
It is called Investment Appraisal.
It is about why someone would attach value to your business.

Capital cost, the amount needed for a venture A.
Cumulative cash flow

$$A + (B-C)/D = Payback period$$

Net present value can also be seen as the current value.

Both NPV and payback period can be used by both, companies and potential investors in determining and making financial decisions.
It is always advisable to establish a basic framework of financial prudence. 

\section{Ratio analysis}

We have fife main types of ratios, the first of which is financial ratios, liquidity ratio, investment ratios, gearing ratios and profitability ratios.

Identify aspects of business components and looking at qualitative processes.

\subsection{Liquidity}

Is all about cash flow.

Investment ratios help investors determining whether an investment is worth investing in.

Liquidity ratios have mainly to do with acid test and current ratio.

\section{Homework}
Identify using a PLC, its financial statements (profit/loss & balancesheet) and calculate five ratios.

Use EPS 8

Password for MIT simulation: A1 
\end{document}
