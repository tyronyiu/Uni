% Preamble
% -----------------------------------------------------------------

\documentclass{article}

% Packages
% -----------------------------------------------------------------

%\usepackage{amsmath}                    % Advanced math typesetting
%\usepackage [utf8] {inputenc}       % Unicode support (Umlauts etc)
%\usepackage [ngerman] {babel}             % Change hyphenationrules
%\usepackage{hyperref}                 % Add a link to your document
%\usepackage{graphicx}                % Add picturesto your document
%\usepackage{listing}      % Source Code formatting and highlighting
\usepackage{enumitem}                  % Used for enumeration-lists

% Document Info
% -----------------------------------------------------------------

\author{Ty Yiu}                                 % The author's name
\title{Hand In One}           % The title of the document
\date{24 September 2018}
\setlength{\parindent}{2em}
\setlength{\parskip}{1em}
 \setlength{\itemsep}{\bigskipamount}
\begin{document}

\maketitle                                        % Generates title 

\clearpage
%
% \tableofcontents{}                  % Generates table of contents 
                                    % from sections and subsections

% Commands
% -----------------------------------------------------------------

%\section{}                                          % Section Title

% --------------------

%\subsection{}                                    % Section Title XS

% --------------------

%\subsubsection{}                                % Section Title XXS

% --------------------

%\paragraph{}                                      % Paragraph Title

% --------------------

%\subparagraph{}                                % Subparagraph Title

% --------------------

%\begin{itemize}                                 % Unenumerated List
%\item ITEM
%\item ITEM
%\item ITEM
%\item ITEM
%\end{itemize}
 
% --------------------

%\begin{enumerate}                         % Enumerated List (1.2.3.)
%\item ITEM
%\item ITEM
%\end{enumerate}

% --------------------

%\begin{enumerate}                         % Enumerated List (1.2.3.)
%\item ITEM                    % With Nested List with Items (a.b.c.)
%\begin{enumerate}
%\item ITEM
%\item ITEM
%\end{enumerate}
%\item ITEM
%\end{enumerate}

% --------------------    

%\textit{}                                              % Italic Text

% --------------------

%\textbf{}                                                 % Bold Text

% --------------------

\section{Lean start-up methodology and the lean canvas}


\textit{What is the lean start-up methodology and the lean canvas? How can it be used and by whom?}


\section{Lean start-up methodology}

As "theleanstartup.com" describes:
The Lean Startup provides a scientific approach to creating and managing startups and get a desired product to customers' hands faster.\par
The Lean Startup method teaches you how to drive a startup-how to steer, when to turn, and when to persevere-and grow a business with maximum acceleration.
It is a principled approach to new product development.

\textbf{Principles}


\begin{itemize}

\item Entrepreneurs are everywhere
  \begin{itemize}
  \item Universal applicability of entrepreneurship  
  \end{itemize}
\item Entrepreneurship is management
  \begin{itemize}
  \item Institutions like start-ups require special management
  \end{itemize}
\item Validated learning
  \begin{itemize}
  \item Learning towards building a sustainable business
  \end{itemize}
\item Innovation accounting
  \begin{itemize}
  \item measure progress $\rightarrow$ modern accounting $\rightarrow$ process optimisation
  \end{itemize}
\item Build-Measure-Learn
  \begin{itemize}
  \item measure, learn to pivot or persist $\rightarrow$ accelerate this feedback loop
  \end{itemize}

\end{itemize}

The lean start-up methodology can be used for process optimisation, these include product development or delivering new products to customers faster.\par
It is a set of principles that apply entrepreneurial thinking to start-up management.


The lean start-up methodology can be used by anyone, as the entrepreneurial mindset is unlimited to its members and the concept of such could be omni-applicable. 


Trying to create a MVP for a business. 


\section{Important aspects for learning:}


\begin{itemize}
  \item A fast fail strategy
\end{itemize}


\section{Lean Canvas}


\textit{Alexander Osterwalder $\leftarrow$ Ash Maury}


Fast, concise and effective startup - Lean startup spirit


\textit{Lean Canvas} describes actionable and entrepreneur focused business plan, focusing on problems and their solutions as well as key metrics and competitive advantages.


It's a layout model for a planning tool, that evaluates strengths and weaknesses of an idea for a product/service.


The lean canvas can be used for outlaying a business model and allows overview over many significant aspects of the business, such as UVP, revenue streams, unfair advantage, etc.
The lean canvas is an idea that has been evolved many times and progressively worked upon, it is not a fix thing.
The lean canvas' idea can be taken and applied to completely new circumstances. 
It can be used in any way of managerial overview situation.


\section{Generic strategies and combined strategies}


Generic strategies and combined strategies - what are they? What can they be used for?


Remember \textit{Concept of competitive advantage}


Three generic strategies for abnormal profit - cost leadership, differentiation, focus


Focus - specialisation and focusing on one segment


Cost-leadership - In business strategy, cost leadership is establishing a competitive advantage by having the lowest cost of operation in the industry.
Cost leadership is often driven by company efficiency, size, scale, scope and cumulative experience (learning curve).


Differentiation - Differentiation is a high risk strategy. 
You cannot differentiate if you do not have market presence.


CAC and CLV can be integrated by considering that dropping the price due to comparative advantage, the cost of acquiring a customer decreases and the customer lifetime value increases due to happier customers as of the lower price.


https://www.whitbread.co.uk/


\section{Combined strategies}


Corporate planning for simultaneous goal execution.


Simultaneous execution hinders focused execution to occur.


Usually, MNC create structures that allow them to conduct simultaneous and in some cases different strategies without necessarily putting each division at risk.

Often, Business create \textit{SBU's} for simultaneous goal execution. 
Geographic structurisation is obviously also a thing, as are the divison into similiarly scaled variables.

\begin{description}
  \item [SBU] Strategic Business Unit
\end{description}

The above strategies provide the foundation upon which organisations can establish both, a comparative advantage, as well as defend against aggresive competition.
Competitive advantage that is easily recreated, offers limited benefits to the company. 

\end{document}
