
% Preamble
% -----------------------------------------------------------------

\documentclass{article}

% Packages
% -----------------------------------------------------------------

%\usepackage{amsmath}                    % Advanced math typesetting
%\usepackage [utf8] {inputenc}       % Unicode support (Umlauts etc)
%\usepackage [ngerman] {babel}             % Change hyphenationrules
%\usepackage{hyperref}                 % Add a link to your document
%\usepackage{graphicx}                % Add picturesto your document
%\usepackage{listing}      % Source Code formatting and highlighting
\usepackage{enumitem}                  % Used for enumeration-lists

% Document Info
% -----------------------------------------------------------------

\author{Ty Yiu}                                 % The author's name
\title{Argumentation}           % The title of the document
\date{27 September 2018}
\begin{document}
\maketitle                                        % Generates title 
%
% \tableofcontents{}                  % Generates table of contents 
                                    % from sections and subsections

% Commands
% -----------------------------------------------------------------

%\section{}                                          % Section Title

% --------------------

%\subsection{}                                    % Section Title XS

% --------------------

%\subsubsection{}                                % Section Title XXS

% --------------------

%\paragraph{}                                      % Paragraph Title

% --------------------

%\subparagraph{}                                % Subparagraph Title

% --------------------

%\begin{itemize}                                 % Unenumerated List
%\item ITEM
%\item ITEM
%\item ITEM
%\item ITEM
%\end{itemize}
 
% --------------------

%\begin{enumerate}                         % Enumerated List (1.2.3.)
%\item ITEM
%\item ITEM
%\end{enumerate}

% --------------------

%\begin{enumerate}                         % Enumerated List (1.2.3.)
%\item ITEM                    % With Nested List with Items (a.b.c.)
%\begin{enumerate}
%\item ITEM
%\item ITEM
%\end{enumerate}
%\item ITEM
%\end{enumerate}

% --------------------    

%\textit{}                                              % Italic Text

% --------------------

%\textbf{}                                                 % Bold Text

% --------------------

\section{Argumentation}

\begin{description}
\item [An argument can be defined as a:] form of reasoning that attempts to establish the truth of one claim (called a conclusion) based on the assumed truth of the evidence in other claims (called premises) provided to support the conclusion.
\end{description}

Can I have a counter example - a strong argument

If you have a valid argument, assuming the premise to be true, there is no other possible solution than the conclusion


An argument has three important characteristics or features, in that it:

\begin{itemize}
  \item is a form of reasoning;
  \item is comprised of claims (sometimes also called statements or assertions);
  \item aims at establishing a conclusion (i.e., one claim)
  \item based on evidence (provided by other claims called premises).
\end{itemize}

\section{The Structure of an Argument}

\subsection{Premise 1.}

When I recently visited the Computer Science Department at the University of Hiroshima I noticed that graduate students and professors there were field testing a new computer chip, whose code name is Chip X.

\subsection{Premise 2.}

I have a copy of the design specifications for Chip X, which shows that it will be several times faster than any chip currently available in the UK.

\subsection{Premise 3.}

Lee Smith, a mutual colleague of ours who was recently an exchange student in the computer science program at the University of Hiroshima and who participated in the field testing of Chip X, will corroborate my account.

\subsection{Conclusion.}

Chip X is currently being developed in Japan.


\section{Argument Structure vs. Argument Strength}

Not all arguments are strong – i.e., not all arguments succeed in establishing their conclusions.
Any form of reasoning will qualify as an argument if it satisfies the three conditions we specified above.
Analyze the following argument and ask yourself whether it is a strong argument – i.e., does it establish its conclusion?

\subsection{Analyze the following argument:}

\subsubsection{Premise:}
The Internet is in public space.

\subsubsection{Conclusion:}
Those who use the Internet should not expect to retain any personal privacy while online.

Is this argument strong – i.e., does it succeed in establishing its conclusion?


\section{Constructing an Argument in Ordinary Language (Prose)}

Consider the following argument that is expressed in prose (or narrative) form:

We must build a national missile defense system (NMD) because without such a system we are vulnerable to nuclear attacks from rogue nations that might arise in the future.
Additionally, several engineers and computer scientists have testified that they can design a computer-guided missile defense system that is effective, safe and reliable.
Furthermore, it is our obligation as British to take whatever measures we can to protect the safety of our citizens.

\subsection{Converting an Argument from Prose into “Standard Form”}

\subsubsection{Premise 1.}
 Without the new National Missile Defense System, the UK is vulnerable to nuclear attacks in the future from "rogue nations.

\subsubsection{Premise 2.}
 Computer scientists and engineers have testified that they can design a computerguided missile defense system that is both safe and reliable.

\subsubsection{Premise 3.}
 The UK. must do whatever is necessary to preserve the military defense of the nation and the safety of its citizens.

\subsubsection{Conclusion.}
Therefore, the UK should build the new National Missile Defense System.

\section{Argument Strength: Valid vs.  Invalid Arguments}

Is the preceding argument involving the new National Missile Defense (NMD) system a strong argument (i.e., is it valid)?
Assume that the premises in the argument are true, does the conclusion necessary follow from them?
Can you imagine a counterexample to the argument?

\subsection{Valid and Invalid Arguments The Counterexample Strategy}

Because you can imagine a counterexample to the NMD argument, it is invalid.
Note that all of the premises in that argument can be assumed true, while the argument’s conclusion could still be imagined to be false.
However, we could revise or reconstruct the NMD argument by adding another premise.

\subsection{The NMD Argument Revised}

\subsubsection{Premise 1.}
Without the new National Missile Defense System, the UK is vulnerable to nuclear attacks in the future from "rogue nations.

\subsubsection{Premise 2.}
Computer scientists and engineers have testified before Congress that they can design a computer-guided missile defense system that is both safe and reliable.

\subsubsection{Premise 3.}
 The UK must do whatever is necessary to preserve the military defense of the nation and the safety of its citizens.

\subsubsection{Premise 4.}
 The national missile defense system is necessary to preserve the defense and safety of the UK and its citizens.

\subsubsection{Conclusion.}
 Therefore, the UK should build the new National Missile Defense System.

Is the revised argument for NMD a valid argument?
Can you imagine a counterexample to it?
If the premises are assumed true, the conclusion must be true, i.e., no counterexample is possible.
So, the (revised) argument is valid.

\section{The Form of a Valid Argument}

A valid argument is valid solely in virtue of its logical form, not its content.

\subsection{An example of a valid logical form is:}

\subsubsection{PREMISE 1.}
Every A is a B.

\subsubsection{PREMISE 2.}
C is an A.

\subsubsection{CONCLUSION.}
C is a B.

No matter what values are substituted for A, B, and C, the argument form is always valid.

\section{Valid and Invalid Arguments Arguments}

\begin{description}

  \item [Valid] The assumed truth of the premises is sufficient to guarantee the conclusion.

  \item [Invalid] Premises (even when true) do not guarantee the conclusion.

  \item [Valid Arguments that are not Sound] An argument can be valid (in virtue of its logical form), but still not succeed in accomplishing its task.
For example, one or more of the (valid) argument’s premises might not be true in the actual world.
In this case the argument would still be valid, but it would not be sound.

\end{description}

    \section{Sound and Unsound Arguments}

\textit{For an argument to be sound, it must be:}
Valid (i.e., the assumed truth of the premises would guarantee the truth of the argument’s conclusion);
The (valid) argument’s premises must also be true in the actual world.


\subsection{Arguments that are Valid and Unsound}

\textit{The following argument is valid, but unsound:}

\subsubsection{PREMISE 1.}
People who own iMac computers are smarter than those who own PCs.

\subsubsection{PREMISE 2.}
My roommate owns an iMac computer.

\subsubsection{PREMISE 3.}
I own a PC.

\subsubsection{CONCLUSION.}
My roommate is smarter than me.


\subsection{Sound Arguments}

Sound arguments are very rare.
The following argument is sound:

\subsubsection{PREMISE 1.}
CEOs of major computer corporations are high-school graduates.

\subsubsection{PREMISE 2.}
Bill Gates was the CEO of a major computer corporation.

\subsubsection{CONCLUSION.}
Bill Gates is a high-school graduate.

\begin{itemize}
  \item Valid Arguments
    \begin{itemize}
      \item Sound
      \item Unsound
    \end{itemize}
\end{itemize}

\begin{itemize}
  \item All the premises are true.
  \item At least one premise is false.
\end{itemize}

    \section{Invalid Arguments: Inductive vs.  Fallacious Reasoning}

An argument is invalid if you can give one counterexample to the argument.
We saw that a counterexample is:

A possible case where the premises can be assumed to be true while, at the same time, the conclusion could be false (Nolt).

    \textit{Invalid arguments will be either:}

inductive, fallacious.

    \subsection{The following argument is invalid because a counter example is possible:}

    \subsubsection{PREMISE 1.}
All CEOs of major United States computer corporations have been United States citizens.

    \subsubsection{PREMISE 2.}
Bill Gates is a United States citizen.

    \subsubsection{CONCLUSION.}
Bill Gates has been a CEO of a major computer corporation in the U.S.

    \section{Inductive Arguments}

    \subsection{An argument is inductive when:}

The conclusion would likely be true when the premises of the argument are assumed to be true.

Even though a counterexample to an inductive argument is possible, the argument’s conclusion would likely be true in the majority of cases where the premises are assumed true.

    \subsection{The following is an example of an inductive argument:}

    \subsubsection{PREMISE 1.}
Most CEOs of computer corporations are college graduates.

    \subsubsection{PREMISE 2.}
Steve Ballmer is the CEO of Microsoft, a computer corporation.

    \subsubsection{CONCLUSION.}
Steve Ballmer is a college graduate.


    \section{Fallacious Arguments}

    \begin{description}
    \item [An argument is fallacious when:] The argument’s conclusion would not likely follow from its premises, even when all of the premises are assumed true.
    \end{description}

Multiple counterexamples to a fallacious argument can be provided.

      \subsection{The following is an example of a fallacious argument:}

      \subsubsection{PREMISE 1.}
Ten percent of people who own iPods also own iMac computers.

      \subsubsection{PREMISE 2.}
My roommate currently owns an iPod.

      \subsubsection{CONCLUSION.}
My roommate also owns an iMac computer.

Invalid Arguments

      \begin{itemize}
        \item Inductive
        \item Conclusion likely follows from assuming the truth of the premises.
        \item Fallacious
        \item Conclusion does not likely follow from assuming the truth of the premises.
        \item A Comprehensive Scheme for Viewing Argument Strength Arguments
        \item Valid
        \item Unsound
        \item Weak Arguments
        \item Invalid
        \item Sound
        \item Inductive
        \item Strong Arguments
        \item Fallacious
        \item Weak Arguments

      \end{itemize}
  \section{Seven-Step Strategy for Evaluating}

  \section{Arguments I: Steps 1-4}

  \subsubsection{Step 1.}
Convert the argument into standard form. (List the premises first, followed by the conclusion.)

\subsubsection{Step 2.}
 Test the argument for its reasoning strength to see whether it is valid or invalid.
(Assume the premises to be true, and ask yourself whether the conclusion must also be true when those premises are assumed true. Is a counterexample to the argument possible?)

\subsubsection{Step 3.}
Is the argument valid?
If yes, go to Step 4.
If no, go to Step 5.

\subsubsection{Step 4.}
Is the (valid) argument also sound?
 That is, are the premises true in the actual world?

\subsubsection{4a.}
If the argument is valid and if all of the premises are true in the actual world, then the argument is also sound.
(To determine truth-values for statements, see Appendix E.)

\subsubsection{4b.}
If the argument is valid, but one or more premises can be shown to be either false or not capable of being verified in the actual world, then argument is unsound.

\subsubsection{Step 5.}
Is the (invalid) argument inductive or fallacious? (How likely will the conclusion be true when the premises are assumed true?)

\subsubsection{5a.}
If the conclusion would likely be true because the premises are assumed true, the argument is inductive.

\subsubsection{5b.}
If the conclusion would not likely be true even when the premises are assumed true, the argument is fallacious.
(Keep in mind that a fallacious argument can be made up of Individual claims that are themselves true in the actual world.)

\subsubsection{Step 6.}
Determine whether the premises in your argument are either true or false.

\subsubsection{Step 7.}
Make an overall assessment of the argument.
That is, describe the argument's strength of reasoning in conjunction with the truth conditions of the argument's premises.
For example, is the argument inductive with all true premises?
Is it inductive with some false premises?
Is it fallacious with a mixture of true and false premises, and so forth?

\medskip

Remember that an inductive argument with premises that are all true is stronger than a valid argument with one or more false premises.)


\section{Scenario}

You are contemplating downloading a software application that is available on a Web site called Sharester, a peer-to-peer (P2P) site set up for file sharing.
Sharester is not officially designated as a “pirate site” because it provides users mainly with access to (freely available) open-source software applications.
However, the site also enables users to download a few proprietary (or copyrighted) software programs.
It turns out that the particular software application you are interested in downloading is proprietary, and there is no reason to believe that the copyright holder of that application has authorized its being freely downloaded by users on this P2P site.
You want to download this application for personal use (only) but you are conflicted about what to do, and you discuss your concerns with a good friend.
Your friend tries to convince you not to download the proprietary software program, using the following rationale: Downloading proprietary software (without permission from the copyright holder) is identical to stealing physical property.
 Stealing physical property is morally wrong.
 Therefore, downloading proprietary so5ware (without permission) is morally wrong.

\medskip

Is the reasoning process used by your friend a good one?
How do we determine that?
Apply the 7 step strategy.


\section{Scenario 2}
Apply the seven-step strategy to evaluate the argument in the following scenario:

A major associa0on represen0ng the music industry in the United States has determined that 4,000 copies of a new album featuring a popular rock group, called DUO, had been illegally downloaded last month.
The cost of this album for those who elect to download it legally from online music stores is \$10 per copy.
So the association concludes that the music company that holds the copyright to this album lost \$40,000 dollars in revenue last month (on that album alone).


\section{Scenario 3}

Determine whether the strength of reasoning used in the argument in the following scenario is valid or invalid.
If it is invalid, does it commit any fallacies?

\medskip

You are engaged in an intense discussion with your friend, Bill, who works in the IT department at your university.
Bill complains that many students are using P2P (peer-to-peer) file-sharing applications on the university’s network to download excessive amounts of unauthorized copyrighted material.
He also claims that the most executive solution to this problem would be to disable student access to all (existing) P2P sites and to prevent students at your institution from se]ng up their own P2P sites for any reason whatsoever (even to include noncopyrighted material).
You convey to Bill your belief that this measure is too drastic.
However, Bill argues that the only way to eliminate unauthorized file sharing among students at your institution is to disable access to all P2P so5ware on the university’s network.  

\end{document}
