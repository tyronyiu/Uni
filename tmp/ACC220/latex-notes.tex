% Preamble
% -----------------------------------------------------------------

\documentclass{article}

% Packages
% -----------------------------------------------------------------

%\usepackage{amsmath}                    % Advanced math typesetting
%\usepackage [utf8] {inputenc}       % Unicode support (Umlauts etc)
%\usepackage [ngerman] {babel}             % Change hyphenationrules
%\usepackage{hyperref}                 % Add a link to your document
%\usepackage{graphicx}                % Add picturesto your document
%\usepackage{listing}      % Source Code formatting and highlighting
%\usepackage{enumitem}                  % Used for enumeration-lists

% Document Info
% -----------------------------------------------------------------

\author{Ty Yiu}                                 % The author's name
\title{Managerial accounting}           % The title of the document
\date{25 September 2018}
\begin{document}
\maketitle                                        % Generates title 
\clearpage
%
% \tableofcontents{}                  % Generates table of contents 
                                    % from sections and subsections

% Commands
% -----------------------------------------------------------------

%\section{}                                          % Section Title

% --------------------

%\subsection{}                                    % Section Title XS

% --------------------

%\subsubsection{}                                % Section Title XXS

% --------------------

%\paragraph{}                                      % Paragraph Title

% --------------------

%\subparagraph{}                                % Subparagraph Title

% --------------------

%\begin{itemize}                                 % Unenumerated List
%\item ITEM
%\item ITEM
%\item ITEM
%\item ITEM
%\end{itemize}
 
% --------------------

%\begin{enumerate}                         % Enumerated List (1.2.3.)
%\item ITEM
%\item ITEM
%\end{enumerate}

% --------------------

%\begin{enumerate}                         % Enumerated List (1.2.3.)
%\item ITEM                    % With Nested List with Items (a.b.c.)
%\begin{enumerate}
%\item ITEM
%\item ITEM
%\end{enumerate}
%\item ITEM
%\end{enumerate}

% --------------------    

%\textit{}                                              % Italic Text

% --------------------

%\textbf{}                                                 % Bold Text

% --------------------

\begin{center}

 \paragraph{Mon Sep 10 09:02:28 BST 2018} 

\end{center}

Managerial accounting, also known as cost accounting, is the process of identifying, measuring, analyzing, interpreting, and communicating information to managers for the pursuit of an organization's goals.

\section{Definition of Managerial Accounting?}

Managerial Accounting is the collection and analysis of financial and non-financial information and communicating it to managers for better decision making processes and increasing share holder wealth.

\section{The importance of Managerial Accounting?}

The importance is implied by the value managerial accounting adds for managers in regards to cost or process. 
Especially, when having only been regarding accounting reports and having to been staying within the standards of accounting.

\section{Managerial Accounting VS. Financial Accounting?}

Managerial Accounting is more concerned with operational reports, which are only distributed within a company.

Financial accounting must comply with various accounting standards, whereas managerial accounting does not have to comply with any standards when information is compiled for internal consumption. Systems.

Management accounting differs from financial accounting. While financial accounting provides information to people inside and, more importantly, people outside the organization, management accounting is mostly aimed at aiding managers inside the organization with decision making.

\clearpage

\section{Decisions requiring managerial accounting information}

\textsf{Figure 1.10 > Page 26 > Chapter 1}

\begin{enumerate}
\item Developing long term plans and strategies 
\item Performance evaluation and control
\item Cost and benefit determination
\item Resource allocation
\end{enumerate}

[[Strategic Management]]

\paragraph{Wed Sep 12 09:00:47 BST 2018}

\textbf{Major tasks of the board:}

\begin{enumerate}
\item Setting the overall direction and strategy for the business
\item Monitoring and controlling the activities of the business 
\item Communicating with shareholders and others connected with the business
\end{enumerate}
   
\textbf{A department structure organised according to business functions}

\textsf{figure 1.1 > page 2 > week 1 Lecture slides}


\textbf{A divisional organisational structure}


\textsf{figure 1.2 > page 3 > week 1 Lecture slides}

Dissecting the business into divisions brings many advantages such as having account of who does what better or having the ability to diversify the same product within different divisions (e.g. European and Asian division adapting products for their market)

\clearpage

\textbf{The changing business environment}

\begin{enumerate}
  \item Increasing sophistication of customers
  \item Development of global economy
    \begin{enumerate}
      \item Creates loss of options for the customers 
      \item Increases options for the developer
    \end{enumerate}
  \item Rapid changes in technology
    \begin{enumerate}
  \item Better management of resources due to enhanced technology application
    \end{enumerate}
  \item [[Deregulation]] of domestic markets
    \begin{enumerate}
      \item Deregulating increases market competition by privatisation 
    \end{enumerate}
  \item Increasing pressure from owners
    \begin{enumerate}
  \item Realistic goals and realistic resources are needed, thus the modern market being very frequent in change of labour needs to embrace patience.
    \end{enumerate}
  \item Increasing volatility of financial market
    \begin{enumerate}
  \item A small drop in exchange rates can really cost a big company a lot of money. 
    \end{enumerate}
\end{enumerate}
    
\textit{Swot analysis - a small review}

\textbf{The strategic management framework}

\begin{verbatim} 

   | Establish mission, vision and objectives     |
   | Undertake a position analysis                |
   | Identify and asses the strategic options     |
   | Select strategic options and formulate plans |
   | Perform, review and control                  |
    
\end{verbatim}

\textbf{The changing business landscape}

    \begin{enumerate}
      \item Growth of the [[Service Sector]]
      \item Emergences of new industries
      \item Growth of e-commerce
      \item Automated manufacturing
      \item [[Lean manufacturing]]
      \item Greater product innovation
      \item Faster response times
    \end{enumerate}

\textbf{Relationship between risk and return}


The higher the wanted return the higher the needed risk. Their relationship is proportional.

\textbf{Main users of accounting information relating to a business}


\textbf{figure 1.7 > page 6 > week 1 Lecture slides}

As shown in the figure, the parties should be considered by the business. Often businesses concentrate too much on e.g. shareholders, causing problems.

\textbf{The qualities that influence the usefulness of accounting information}


Qualities of different people should be applied to the correct jobs. The figure shows how that could possibly look like.

\textbf{Relationship between costs and the value of providing additional management accounting information}


If you start to collect the necessary information, it costs more and more until a law of diminishing returns is applied. 
E.g. paying an expensive firm to write reports on market research maybe doesn't make sense in circumstances. Tools like AI or similar don't add as much marginal benefit as expected or the staff to operate/optimise the tools isn't correctly trained or not present.

\textbf{The management accounting information system}


Identifying data, recorded, then analysed until they can be used to write insightful reports

\textbf{The four phases in the development accounting}


Not only is thought of what and how to produce, but also about what shareholders hold value in or cost controlling or waste reduction are used.

\textbf{Management and financial accounting}

\begin{enumerate}
  \item Nature of the reports produced
    \begin{enumerate}
  \item Managerial is more specific and financial accounting has standards. 
    \end{enumerate}
  \item Level of detail
  \item Existence of regulations
  \item Reporting interval
    \begin{enumerate}
  \item Managerial accounting doesn't have reporting intervals, the reports are done when needed.
    \end{enumerate}
  \item Time orientation
    \begin{enumerate}
  \item Management accounting is more about what to do in the future than what happened in the past. 
    \end{enumerate}
  \item Range and quality of information
    \begin{enumerate}
  \item Customer satisfaction or similar is also relevant information for management accounting
    \end{enumerate}
\end{enumerate} 

[[Review Exercises Chapter 1]]

\clearpage

\paragraph{Mon Sep 17 09:16:52 BST 2018}

[[Week 2 additional Exercises]]

\textbf{Audit 2023 Audits technology fit for the future}

https://www.youtube.com/watch?v=AAyjbAJEwfM

\textbf{Relevant and Irrelevant costs}

What data should influence your decision making process? - Divide between relevant and irrelevant information. Also beware of historical data to not influence present issues. 

In a business, only relevant cost should be considered for decision making purposes and irrelevant costs should be ignored.

Relevant costs should have three attributes:
    \begin{enumerate}
      \item It should relate to the future
      \item Cost should vary with the decision
      \item Maximise wealth and relate to the objective of the business  
    \end{enumerate}
   
If the cost is exactly the same in two scenarios of outcomes of decisions, the cost is irrelevant.

\textbf{Definition of Cost}


The amount of resources, usually measured in monetary terms, sacrificed to achieve a particular objective.

\textbf{Two Types of cost:}

    \begin{enumerate}
      \item Historic cost (A cost already incurred)
      \item opportunity cost (The value of an opportunity forgone)
    \end{enumerate}
   
\textbf{Relevant Costs:}

    \begin{enumerate}
      \item Opportunity cost 
      \item Future outlay costs (Those that vary with the decision)
    \end{enumerate}
   
\textbf{Irrelevant Costs:}

\begin{enumerate}
  \item Historic costs
  \item Future outlay costs 
\end{enumerate}

There is no additional cost for a business if e.g. employees are allocated to another task, as they are already being paid anyways.
E.G. A company asks for your technicians for a side project, they should be accounted for at least what they are being paid in order to not incur extra cost.

\textbf{Irregular and Regular use of raw materials}


Jay Ltd is bidding for a contract that will require 800 units of Alpha.
This material is currently held and information is as follows:
What is the minimum price for Alpha for inclusion in the contract bid?

\textbf{Irregular use of Alpha}


800 ×£12 = £9,600


\textbf{Regular use of Alpha}


800 ×£14 = £11,200


If existing raw materials are regularly used in business, we use replacement cost (market buying price) 


If any item is not normally/regularly used, we use sale price (resale value).


committed cost is irrelevant as it has already been accounted for.


If the labour has spare capacity, there is no additional cost as the labour is already being paid for by the business employing that labour.


\paragraph{Wed Sep 19 09:00:56 BST 2018}


[[Activity 2.6]]

[[Chapter 2 Self-assessment question 2.1]]


[[Cost-Volume Profit Analysis]]


\clearpage

\paragraph{Mon Sep 24 09:05:52 BST 2018}


[[RELEVANT COSTING: ADDITIONAL EXERCISES]]


3,000 - not in use - Resale value of 1,000 - Replacement cost of 5,000

Could subsitute for Z(6,000) + 2,000 
either 6,000 or (3,000 + 2,000)

Relevant Cost for X 

\textbf{Relevant Costs:}

\begin{enumerate}
  \item Opportunity cost 
  \item Future outlay costs (Those that vary with the decision)
\end{enumerate}

\begin{verbatim}

Opportunity cost (to Z) | 5,000
Opportunity cost (sell) | 1,000

\end{verbatim}

To start a project, the business can take material already in stock. 

\begin{verbatim}

Materials                   Project          Market    
=========                   =======          ======
in use (replacement cost)                    market price
not in use (resale value)   Cost Ex          

Option 1: Sell for 1,000
Option 2: Substitute Z
          Saving: 6,000
          Proc. Cost: 2,000
          Net savings: 4,000

\end{verbatim}

Component X in use - 120 units in stock - 
Component Y no use - 300 units in stock - resale value of 18pp (18*300=5,400)
Component Z in use - 20 units in stock - 

\paragraph{Project 1: }

100 units of X - we have it - 100 * 12 = 1,200
400 units of Y - buy 100 more - 22 * 100 = 2,200
100 units of Z - buy 70 more - 70 * 45 = 3,150

Total relevant cost: 2,200 + 3,150 + 1,200 = 6,550 
\begin{verbatim}
- Component A 3 per unit
- Component B 1 per unit
- Component C 2 per unit
\end{verbatim} 

10 * Peeping Tom

Development cost of 220,000 (sunk cost)

For Component B, because it is not in use and there is a disposing cost as well as the MOQ, we have to buy 6, even though we only need 4, and then need to dispose the excess inventory due to its zero-value.

\textbf{The disposing cost is irrelevant.}

Because if the project would not be undertaken, the disposing cost would incur as well, thus the cost appears in both decisions (doing the project or not), and the cost is irrelevant.

\clearpage

\begin{verbatim}

Component A:              Component B:                Component C:
============              ============                ============
widely in use             no longer used              in constant use
15 in stock               6 in stock                  none in stock
                          2,200 each                  20 unit contract about to be placed
1,800 each                MOQ of 6 units              1,700 per units 
+ 5%                      Disposing cost (300p/unit)  for orders above 30, 10% off
= 1890                    

Material Cost             Material Cost               Material Cost

10 * 3 = 30               10 * 1 = 10                 10 * 2 = 20
                                                      
Replacement cost          6 * 2,200                   20 * 1,700 = 34,000
                          6 * 0                       40 * 170 = 6,800
                                                         
                          Total: 13,200               Total: 27,200
                          
\end{verbatim}

\end{document}
