
\documentclass[12pt]{article}
\usepackage{cite}
\usepackage{caption}
\usepackage{subcaption}
\usepackage[hidelinks]{hyperref} %Makes Table of Contents clickable

\begin{document}
% Document Configuration
% ----------------------
\title{Reflective Essay}
\author{Ty Yiu}
\date{2019-12-22}
\maketitle
\tableofcontents
\pagebreak

\section*{abstract}
\addcontentsline{toc}{section}{Abstract}
Having known not much about professional sports leagues, organisational
structures or anything remotely connected to traditional sports, this course
extraordinarily increased my knowledgeability in that specific area. 
Mix some biology or chemistry into the material or neurological / psychological
explanations and even a non-jock could get interested in sports management. More
importantly, seeing \textit{sports} as an analogy for \textit{stress situations} or
situations requiring \textit{high performance}, makes the name a little confusing, but
at the same time more relevant to more individuals. 
Luckily for me, that is exactly what the course and instructor did.

\pagebreak

\section{Introduction}
\begin{quotation}
\textit{Reflective practice is the process of continuous learning enabled
through questioning one self's past actions.}
\end{quotation}

Luckily, reflective practice has been a font technique of mine for years.
For long, I have tried incorporating meditation, stoicism, elements of zen and
buddhism, feynman technique, and controlled breathing, into reflective practice. 
To me, it is a central element of self-optimisation. Reading loads of material
in the genre of philosophy, reflection became a natural part of my habit list.
Specifically, this course expanded my horizon as to grasp and transfer
conclusions forged from reflection. Following a Wittgenstein' mentality of
knowledge transfer, the course and instructor succeeded at sharing valuable
insights.

Assessing one self's performance in past situations can reveal insights into why
performance was as it was, considering given circumstances. Thus, conclusions
can be induced into what could be the source of unexpected underperformance and
how to avert re-occurrence. 

The more factors one can include in such analysis of reflective practice, the
more likely one may be to conclude more precise strategies objectifying learning.

For the participated in class \textit{"sports management"}, reflection was
crucial as to the aim of increasing capability for high performance situations.
These high performance situations should be subject to reflective practice for
optimisation of potentially identified shortcomings. Furthermore, a log shall be
maintained capturing demonstration of learning manifested through reflective
practice.

The three main deliverables, the assignments are a guideline for structure, as
well as outstanding moments from lectures. 

\section{Employed reflective practice}
Similarly to Gibbs's model of reflective practice \cite{gibbs:reflectiveModel},
emotions will be considered for a trigger of reflective practice, though shall
not be the sole origin of such. Thus, the goal is to be able to even reflect
upon apathetic situations as to wonder why apathy may have manifested itself.
The goal of this is, to tailor reflective practice more to individual
personality traits, as one may easily be stimulated or may be avoiding
stimulation.

Stress is a very significant factor when it comes to high performance
capabilities. A recently undertaken personality trait test revealed a 9/10
stress resistance of the \textit{performance model} base-lined to high
performance business management environments. The \textit{triune brain}
incorporates stress into the \textit{two-brain theory} of human behaviouristics.
Stress is less of a personal stimulator as to stay in mind. On the contrary, a
personally identified shortcoming is conflict avoidance. This has been revealed
through analysis of apathetic behaviour in situations of unlikely display of
such. The conclusion made is a decreased level of assertiveness or lack of trust
in own ability.

A situation reflected upon in the log is the assertive and questioning nature of
the instructor requesting confirmation of a student stated opinion. Such
behaviour is strategization of testing the student supporting their own stance
under pressure of it being questioned. 

The same way, the analysis of the heart rate baseline variance is very
interesting to myself in the sense of whether or not one's own image of one is
skewed. Such could be the case considering the existence of unconscious
incompetence. Specifically, my own classification of capability, skill or
resistance may be completely biased. Such is a much more interesting question to
me compared to why one has performed a certain way, as underlying biases can
even skew that analysis heavily. Hence, an outside entity may be required,
providing a probably less biased perspective on one's performance. Thus,
feedback is essential to reflection. It can thus be said, that
\textit{introspective reflection} can only reveal so much and limited by nature.
The in class discussed \textit{dialogical other} may be a broader approach,
though also less precise and can become skewed in trade off for accuracy in
repetition, due to biases incorporating in the dialogical other.

\section{Reflections}

A situation from class, in fact, the very first class, was rather outstanding to
me. Such was, the task of ordering by city, that students are living in. The
task seemed fairly easy but turned out not to be such at all. Ignorance to the
task displayed by others or plain simple incompetence throughout the course
leaves me to wonder of showing similar behaviour. This is to say, without direct
comparisons being made with others, considering clear differences in personality
or individual factors such as past. Another situation very similar, was one of
the very last lectures, in which the class was tasked with passing a ball
through the air as many times as possible in a given time period. Naturally, the
students failed miserably at coordinated teamwork and clear employment of
individualistic behaviour taking over. A high performance situation without
coordination or prior determined options of action can easily become chaotic and
result in significant drops in collective performance. Students could clearly be
categorised into being drawn to either pole, panic or resilience. 

It would have really interest me to gain insights into the heart rate variance
analysis, to see whether the calm self perspective is the case or a mere
illusion to unconscious incompetence. I would want to be able to control my
bodily reaction to stress situations and minimise its impact. A personal
investigation of stress analysis to be conducted is measuring the bpm before,
during and after high performance situations and to actively try to maintain
being as close to base line 60bpm as possible. The data from the presentation
held as the individual assignment would be golden for analysis. 

Speaking of which, the individual presentation did actually result in a bigger
physiological response to stress than expected. Usually, as can be clearer seen
in the group presentation, there is little apparent stress to my behaviour. My
personal guess into why, is the lack of isolation prior to the presentation, but
instead embracing the moment, chatting with others instead of focussing,
breathing and staying calm.

The psychological character trait analysis classified my level of bravery and
preference of common and thus comfort environments as low, leaving me again to
wonder about the actual level of performance and again outlining my weaknesses.

I compare it to an exam room with extremely nervous teenage girls; there will be
little chance of one being able to not be nervous when absorbing the
surroundings. Thus, for the first time, I identified isolation as a potential
necessity to stress resilience, or at least considering the option.

The by myself displayed nervousness very much followed Gibb's
model \cite{gibbs:reflectiveModel}. The presence of strong emotion is rather
uncommon to me and left me to think about the relevance of Gibb's model, as it
has prior been seen as less relevant to myself.
 
During the individual presentation, I constantly found myself thinking to not
hide myself behind the counter and monitor like everybody would probably do
instinctively, considering the high stress circumstance. Cyclically, I had spare
capacity to realise that I've been subconsciously wandering back to the side
of the camera field of view, instead of staying alone in dead center. Such
behaviour I also consider mentioned subconscious conflict avoidance.

One thing I would have liked, is feedback on the presentation, prior to having
hand in the first assignment. I am unsure about my performance and fear skewed
perception and would have liked to traced differences.

In contrast to the individual presentation, the group assignment felt
comfortable. The group constellation was common and trust in capabilities of
partners was given. The performance of each member was predictable and
acceptedly so and didn't vary greatly in actuality. Much preparation and
fractioned responsibility was key for achieving so, in a subject matter about
which none of the group's members were particularly knowledgeable of.

Collective reflective practice performed and analysis of each other's
performance post to action maximised learning. I believe that the professional
approach and formality required for the presentation was outstandingly good for
introducing less common elements into Hult presentations. It was different than
other assignments, as precision was of the essence.


Although having hated the effort input required for the collective reflective
practice undertaken during classes, the value could be argued for. Having to
change seats and getting up on first glance seems annoying, though brings often
underconsidered benefits such as blood circulation and the \textit{Zeigarnik
Effect}.

Another aspect I disliked due to efficacy, was the chosen method of having
students answer a posed question. The implemented technique left me to stay
inside my comfort zone, without having to fear a probable scenario of needing to
answer on the spot. That I realised over time, made me sometimes participate
less in class in terms of input. The argument for that, is the behaviour
displayed when the guest lecturers were around. Much more discussion was going
on in regards to the number and variety of participating students. Furthermore,
only a few students were forming the majority of participation. 


\section{Conclusion}
The class posed fewer personal challenges than it made me wonder and reflect
upon my actions during the course. Post-observationally, I displayed much higher
and prolonged self awareness during the course than during any other. The focus
on one self and the importance put onto reflection impacted me much more than
the perceived material of the course. My opposition to sports management
transformed into thinking of sports management more of as \textit{stress} or
\textit{high performance} management. The lessons learned from the course posed
persistent and remaining questions and support inquiry, very much unlike the
original perception of the course. 
The course has concluded leaving much room for further progression and having
laid a solid foundation for the professional high performance environment one
could find oneself in.

\pagebreak

\begin{itemize}
    \item First class:
        \begin{itemize}
            \item Sort by region
            \item absolute disaster - uncoordinated
            \item little groups forming
            \item less about sports than thought? 
        \end{itemize}
    \item Group presentation:
        \begin{itemize}
            \item good group; know the team
            \item assignment not really posing a threat
            \item sport affine and non-sport affine members
            \item creativity! -Giancarlo 
        \end{itemize}
    \item General:
        \begin{itemize}
            \item a lot of people are unconsciously incompetent and or ignorant
            \item maybe I should get more involved in collective reflective
                practice
            \item maybe I rely too much on introspective reflection?
            \item how can I use what I've learned to eliminate shortcomings?
            \item interesting approach of sports management being analogous for
                high performance circumstance
            \item I can treat most findings not as sport specific, but high
                performance specific -> much more relevant to me
            \item physical performance less interesting than psychological
            \item people being "exposed" by instructor -> very funny
            \item let's see when it's my turn -> never came
            \item maybe it didnt because I subconsciously tried to avoid it?
            \item instructor really good at making students be sure of what they
                say
        \end{itemize}
    \item Guest lecturers:
        \begin{itemize}
            \item enjoyed first one much more than second; too much "sport"
            \item participation and focused dropped significantly for second
                speaker 
            \item interesting similarities in behaviour between the instructor
                and guest lecturers
            \item relevance of first lecture perceived as higher
        \end{itemize}
    \item Individual presentation:
        \begin{itemize}
            \item O.K. prepared
            \item annoying pre-presentation phase with other students
            \item why was I so nervous? - maybe bc of the others?
            \item Did I do well? Did I stutter? How about movement and
                presentation (body posture etc)
        \end{itemize}
\end{itemize}

\pagebreak

\bibliography{Assignment3}{}
\bibliographystyle{apalike}

\end{document}

[*Consciousness is one of the biggest factors determining
success.*]( http://bit.ly/38tQdL6)


